
\section{Runtime Analysis}
\label{section:analysis}


\subsection{Pair interaction calculations}
\label{section:pairinteraction}
\NAMD\ supportes the calculation of interaction energy calculations between 
two groups of atoms.  When enabled, pair interaction information will be
calculated and printed in the standard output file on its own line at the
same frequency as energy output.  The format of the line is
{\tt PAIR INTERACTION: STEP: {\it step} VDW\_FORCE: {\it fx fy fz} 
ELECT\_FORCE: {\it fx fy fz}}.
The displayed force is the force on atoms in group 1 and is units of 
kcal/mol/\AA. 

For trajectory analysis the 
recommended way to use this set of options is to use the NAMD Tcl scripting 
interface as described in Sec.~\ref{section:tclscripting} to run for
0 steps, so that NAMD prints the energy without performing any dynamics.

\begin{itemize}

\item
\NAMDCONFWDEF{pairInteraction}{is pair interaction calculation active?}
{{\tt on} or {\tt off}}{{\tt off}}
{Specifies whether pair interaction calculation is active.}

\item
\NAMDCONFWDEF{pairInteractionFile}{PDB file containing pair interaction flags}
{UNIX filename}{{\tt coordinates}}
{PDB file to specify atoms to use for pair interaction calculations.  If 
this parameter is not specified, then the PDB file containing initial 
coordinates specified by {\tt coordinates} is used.}

\item
\NAMDCONFWDEF{pairInteractionCol}{column of PDB file containing pair 
interaction flags}{{\tt X}, {\tt Y}, {\tt Z}, {\tt O}, or {\tt B}}{{\tt B}}
{
Column of the PDB file to specify which atoms to use for pair interaction
calculations.  This parameter may specify any of the floating point
fields of the PDB file, either X, Y, Z, occupancy, or beta-coupling
(temperature-coupling).  
}

\item
\NAMDCONFWDEF{pairInteractionSelf}{compute within-group interactions instead of
bewteen groups}{{\tt on} or {\tt off}}{{\tt off}}
{
When active, NAMD will compute bonded and nonbonded interactions only for atoms 
within group 1.  
}
 
\item
\NAMDCONF{pairInteractionGroup1}{Flag to indicate atoms in
group 1?}{integer}{}

\item
\NAMDCONF{pairInteractionGroup2}{Flag to indicate atoms in
group 2?}{integer}{}
{
These options are used to indicate which atoms belong to each interaction 
group.  Atoms with a value in the column specified by {\tt pairInteractionCol} 
equal to {\tt pairInteractionGroup1} will be assigned to group 1; likewise
for group 2.
}
\end{itemize}

\subsection{Pressure profile calculations}
\NAMD\ supports the calculation of lateral pressure profiles as a function of
the z-coordinate in the system.  The algorithm is based on that of 
Lindahl and Edholm (JCP 2000), with modifications to enable Ewald sums based on
Sonne et al (JCP 122, 2005). 

The simulation space is partitioned into slabs, and half the virial
due to the interaction between two particles is assigned to each
of the slabs containing the particles.  This amounts to employing
the Harasima contour, rather than the Irving-Kirkwood contour, as
was done in \NAMD\ 2.5.  The diagonal components of the pressure
tensor for each slab, averaged over all timesteps since the previous
output, are recorded in the \NAMD\ output file.  The
units of pressure are the same as in the regular \NAMD\ pressure
output; i.e., bar.

The total virial contains contributions from up to four components: 
kinetic energy, bonded interactions, nonbonded interactions, and an Ewald
sum.  All but the Ewald sums are computed online during a normal simulation
run (this is a change from \NAMD\ 2.5, when nonbonded contributions to the
Ewald sum were always computed offline).  If the simulations are performed
using PME, the Ewald contribution should be estimated using a separate,
offline calculation based on the saved trajectory files.  The nonbonded
contribution using a cutoff different from the one used in the simulation
may also be computed offline in the same fashion as for Ewald, if desired.

Pressure profile calculations may be performed in either constant volume 
or constant pressure conditions.  If constant pressure is enabled, the
slabs thickness will be rescaled along with the unit cell; the dcdUnitCell
option will also be switched on so that unit cell information is stored in
the trajectory file.

\NAMD\ 2.6 now reports the lateral pressure partitioned by interaction type.
Three groups are reported: kinetic + rigid bond restraints (referred to as 
``internal", bonded, and nonbonded.  If Ewald pressure profile calculations
are active, the Ewald contribution is reported in the nonbonded section, and
no other contributions are reported.

\NAMD\ 2.6 also permits the pressure profile to be partitioned by atom type.
Up to 15 atom groups may be assigned, and individual contribution of each
group (for the ``internal" pressures) and the pairwise contributions of
interactions within and between groups (for the nonbonded and bonded pressures)
are reported in the output file.

\begin{itemize}
\item
\NAMDCONFWDEF{pressureProfile}{compute pressure profile}{{\tt on} or {\tt off}}{{\tt off}}
{
When active, NAMD will compute kinetic, bonded and nonbonded (but not 
reciprocal space) contributions to the 
pressure profile.  Results will be recorded in the \NAMD\ output file
in lines with the format
{\tt PRESSUREPROFILE: ts Axx Ayy Azz Bxx Byy Bzz ... }, where {\tt ts} is the
timestep, followed by the three diagonal components of the pressure tensor 
in the first
slab (the slab with lowest {\it z}), then the next lowest slab, and so forth.
The output will reflect the pressure profile averaged over all the steps since
the last output.  

\NAMD\ also reports kinetic, bonded and nonbonded contributions separately,
using the same format as the total pressure, but on lines beginning with
{\tt PPROFILEINTERNAL}, {\tt PPROFILEBONDED}, and {\tt PPROFILENONBONDED}.
}
\item
\NAMDCONFWDEF{pressureProfileSlabs}{Number of slabs in the spatial partition}{Positive integer}{10}{
\NAMD\ divides the entire periodic cell into horizontal slabs of equal 
thickness; {\tt pressureProfileSlabs} specifies the number of such slabs.
}

\item
\NAMDCONFWDEF{pressureProfileFreq}{How often to output pressure profile
data}{Positive integer}{1}{
Specifies the number of timesteps between output of pressure profile data.
}

\item
\NAMDCONFWDEF{pressureProfileEwald}{Enable pressure profile Ewald sums}{{\tt on} or {\tt off}}{{\tt off}}{
When enabled, only the Ewald contribution to the pressure profile will be
computed.  For trajectory analysis the 
recommended way to use this option is to use the \NAMD\ Tcl scripting 
interface as described in Sec.~\ref{section:tclscripting} to run for
0 steps, so that NAMD prints the pressure profile without performing any 
dynamics.

The Ewald sum method is as described in Sonne et al. (JCP 122, 2005).  The
number of $k$ vectors to use along each periodic cell dimension is specified
by the {\tt pressureProfileEwald}$n$ parameters described below.
}
\item
\NAMDCONFWDEF{pressureProfileEwaldX}{Ewald grid size along X}
{Positive integer}{10}{}
\item
\NAMDCONFWDEF{pressureProfileEwaldY}{Ewald grid size along Y}
{Positive integer}{10}{}
\item
\NAMDCONFWDEF{pressureProfileEwaldZ}{Ewald grid size along Z}
{Positive integer}{10}{}

\item
\NAMDCONFWDEF{pressureProfileAtomTypes}{Number of atom type partitions}{Positive integer}{1}{
If {\tt pressureProfileAtomTypes} is greater than 1, \NAMD\ will calculate
the separate contributions of each type of atom to the internal, bonded, 
nonbonded, and total pressure.  In the case of the internal contribution,
there will be $n$ pressure profile data sets reported on each 
{\tt PPROFILEINTERNAL} line, where $n$ is the number of atom types. All the 
partial pressures for atom type 1 will be followed by those for atom type 2,
and so forth.  The other three pressure profile reports will contain 
$n(n+1)/2$ data sets.  For example, if there are $n=3$ atom types, the
six data sets arising from the three inter-partition and the three 
intra-partition interactions will be reported in the following order:
1--1, 1--2, 1--3, 2--2, 2--3, 3--3.  The total pressure profile, reported
on the {\tt PRESSUREPROFILE} line, will contain the internal contributions 
in the data sets corresponding to 1--1, 2--2, etc.  
}

\item
\NAMDCONFWDEF{pressureProfileAtomTypesFile}{Atom type partition assignments}
{PDB file}{coordinate file}{
If {\tt pressureProfileAtomTypes} is greater than 1, NAMD will assign
atoms to types based on the corresponding value in {\tt pressureProfileAtomTypesCol}.  The type for each atom must be strictly less than 
{\tt pressureProfileAtomTypes}!}

\item
\NAMDCONFWDEF{pressureProfileAtomTypesCol}{{\tt pressureProfileAtomTypesFile}
PDB column}{PDB file}{B}{}

\end{itemize}

Here is an example snippet from a \NAMD\ input that can be used to compute
the Ewald component of the pressure profile.  It assumes that the 
coordinates were saved in the dcd file {\tt pp03.dcd}) every 500 timesteps.  
\begin{verbatim}

Pme             on
PmeGridSizeX    64
PmeGridSizeY    64
PmeGridSizeZ    64

exclude         scaled1-4
oneFourScaling  1.0

switching on
switchdist      9
cutoff          10
pairlistdist    11

pressureProfile        on
pressureProfileSlabs   30
pressureProfileFreq    100
pressureProfileAtomTypes 6
pressureProfileAtomTypesFile atomtypes.pdb
pressureProfileEwald  on
pressureProfileEwaldX  16
pressureProfileEwaldY  16
pressureProfileEwaldZ  16

set ts 0
firstTimestep $ts

coorfile open dcd pp03.dcd
while { [coorfile read] != -1 } {
  incr ts 500
  firstTimestep $ts
  run 0
}
coorfile close
\end{verbatim}


